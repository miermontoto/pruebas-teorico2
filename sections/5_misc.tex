\chapter{Preguntas de los enunciados}
\section{Primera entrega}
\subsection{Posibles consecuencias ante un fallo}
\textit{Si este software se tuviera que desplegar en un servicio web accesible a los ciudadanos,
¿qué consecuencias tendría que dicho servicio fallara y no estuviera disponible?}

El fallo de un software de estas características, ya sea por un error en el cálculo de los impuestos, caída
del servicio o cualquier otro problema que afecte al funcionamiento del sistema, puede tener consecuencias
catastróficas para los ciudadanos, ya que el cálculo de los impuestos es un proceso que afecta directamente
a la economía tanto del individuo como del estado.

En caso de fallo, la ciudadanía podría verse afectada por sanciones económicas por parte de la administración
sin justificación ninguna o incluso por la pérdida de dinero en concepto de impuestos que no se deberían haber
pagado. Por otro lado, el estado podría verse afectado por la pérdida de ingresos en concepto de impuestos que
no se han cobrado, lo que podría afectar a la economía del país.

\subsection{Noticias relevantes}
El fallo de software relacionado con administraciones públicas no es ficción, y de hecho es
más que frecuente. A continuación se citan algunas noticias que dejan en evidencia este hecho:
\begin{itemize}
	\item <<El 61\% de los usuarios han tenido problemas al usar las webs o apps de administraciones públicas>>~\cite{newtral}
	\item <<Fallos en los programas y sistemas caídos: programas desactualizados lastran la actividad en Empleo>>~\cite{abc}
	\item <<Sigue sin funcionar>>~\cite{elpais}
\end{itemize}

\newpage{}
\section{Segunda entrega: Aspectos deontológicos}
\textit{¿Qué aspectos deberíamos tener en cuenta como profesionales, desde un punto de vista ético
y deontológico, al diseñar el modo en que se tratan este tipo de ficheros con datos personales y
económicos de ciudadanos por parte de los usuarios del software?}

Como profesionales, debemos tener en cuenta que los datos personales y económicos de los ciudadanos
son información sensible y privada, por lo que debemos garantizar la confidencialidad y seguridad
de los mismos. Para ello, debemos cumplir con la normativa vigente en materia de protección de datos
personales, como el RGPD, y garantizar que los datos se tratan de forma lícita, leal y transparente.

Además, debemos garantizar que los datos se tratan de forma segura, evitando accesos no autorizados
y protegiendo los datos frente a posibles pérdidas o daños. Implementar medidas de seguridad, como
el cifrado de los datos, el control de accesos o la realización de copias, es la mejor estrategia
para garantizar la seguridad de los datos.
