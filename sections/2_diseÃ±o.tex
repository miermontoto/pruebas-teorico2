\chapter{Diseño de pruebas}
Según la descripción del sistema que otorga el enunciado, se identifican los siguientes conjuntos de entradas
que definen las clases de equivalencia. Para definir las clases de equivalencia se utiliza una sintaxis
matemática, donde se define el conjunto de valores que pertenecen a las mismas.

A continuación se definen las clases de equivalencia para las entradas y salidas del sistema:

\section{Clases de equivalencia: entrada}
\subsection{Base imponible ($BI$)}
La base imponible se toma como la suma de ingresos de todos los empleadores
(en caso de tener más de uno) y se define por tramos según el enunciado.
A esta suma ``total'' se le denomina $BI$.

Puesto que puede haber un número (teóricamente) \textit{infinito} de pagadores,
se acumulan todos los ingresos de los empleadores \textit{secundarios} (es decir, que no
son los principales) en una variable $BI_{o}$, mientras que el principal se define como $BI_{p}$.

\begin{enumerate}
	\item $BI \in [0, 9000)$€
	\item $BI \in [9000, 12450)$€
	\item $BI \in [12450, 20200)$€
	\item $BI \in [20200, 35200)$€
	\item $BI \in [35200, 60000)$€
	\item $BI \in [60000, 300000)$€
	\item $BI \in [300000, \infty)$€
	\item $BI \notin \mathbb{R}, BI < 0$€ (\textcolor{red}{Inválido})
\end{enumerate}

\subsection{Retenciones ($R$)}
Donde $I$ es el cálculo de impuestos a pagar.

\begin{enumerate}
	\item $R \in [0, I)$€
	\item $R \in (I, \infty)$€
	\item $R = I$€
	\item $R \notin \mathbb{R}, R < 0$€ (\textcolor{red}{Inválido})
\end{enumerate}

\subsection{Número de empleadores ($n_{E}$)}
\begin{enumerate}
	\item $n_{E} = 0$ (\textcolor{orange}{no puede tener $BI$})
	\item $n_{E} = 1$
	\item $n_{E} \in [2, \infty)$
	\item $n_{E} \notin \mathbb{N}, n_{E} < 1$ (\textcolor{red}{Inválido})
\end{enumerate}

\subsection{Préstamos hipotecarios ($P_{H}$)}
En este caso, se define $P_{H}$ como la cantidad de dinero que se ha pagado
en concepto de préstamos hipotecarios en viviendas habituales anteriores a 2013.

$60266$€ es el valor por encima del cual se dejaría de aplicar la deducción por vivienda habitual,
que resulta del siguiente cálculo: $$\frac{9040}{0.15} = 60266.66666\dots$$

\begin{enumerate}
	\item $P_{H} = 0$€
	\item $P_{H} \in (0, 60266]$€
	\item $P_{H} \in (60266, \infty)$€
	\item $P_{H} \notin \mathbb{R}, P_{H} < 0$€ (\textcolor{red}{Inválido})
\end{enumerate}

El valor máximo de la deducción es 9040€, lo que no significa que por encima de dicho valor no se aplique
la deducción, sino que el valor de la deducción no aumenta por encima de dicho valor.

\newpage{}
\section{Clases de equivalencia: salida}
\subsection{Obligatoriedad de la declaración}
\begin{itemize}
	\item Obligatorio
	\item No obligatorio
\end{itemize}

\subsection{Gravamen máximo}
Puesto que el cálculo del gravamen se realiza por tramos, se definen los siguientes
valores posibles para el gravamen máximo.

\begin{itemize}
	\item 0\%
	\item 19\%
	\item 24\%
	\item 30\%
	\item 37\%
	\item 45\%
	\item 47\%
\end{itemize}

\subsection{Cálculo (resultado)}
El cálculo de la declaración se realiza en base a las retenciones y la base imponible por \textit{tramos}
tal y como define el enunciado, por lo que la salida del cálculo final dependerá de los valores
de las entradas y no se puede definir una clase de equivalencia para la salida, al exisitir un número
infinito de posibles combinaciones de valores de entrada.

Sin embargo, se definen las tres siguientes opciones, englobando en ellas todos los posibles valores numéricos:
\begin{itemize}
	\item A pagar
	\item A devolver
	\item Sin cambios
\end{itemize}

Esta salida tiene especial conexión con la entrada $R$, ya que el cálculo de la declaración
depende directamente de la cantidad de retenciones que se han realizado. Al cálculo final de esta salida
deberían aplicarse las deducciones correspondientes (en este enunciado solo se considera $P_{H}$).
