\chapter{Clases de equivalencia}
Según la descripción del sistema que otorga el enunciado, se identifican los siguientes conjuntos de entradas
que definen las clases de equivalencia. Para definir las clases de equivalencia se utiliza una sintaxis
matemática, donde se define el conjunto de valores que pertenecen a las mismas.

A continuación se definen las clases de equivalencia para las entradas y salidas del sistema:

\section{Entradas}
\subsection{Base imponible ($BI$)}
La base imponible se toma como la suma de ingresos de todos los empleadores
(en caso de tener más de uno) y se define por tramos según el enunciado.
A esta suma ``total'' se le denomina $BI$.

Puesto que puede haber un número (teóricamente) \textit{infinito} de pagadores,
se acumulan todos los ingresos de los empleadores \textit{secundarios} (es decir, que no
son los principales) en una variable $BI_{o}$, mientras que el principal se define como $BI_{p}$.

\begin{enumerate}
	\item $BI \in [0, 9000)$€
	\item $BI \in [9000, 12450)$€
	\item $BI \in [12450, 20200)$€
	\item $BI \in [20200, 35200)$€
	\item $BI \in [35200, 60000)$€
	\item $BI \in [60000, 300000)$€
	\item $BI \in [300000, \infty)$€
	\item $BI \notin \mathbb{R}, BI < 0$€ (\textcolor{red}{Inválido})
\end{enumerate}


\subsection{Retenciones ($R$)}
Donde $I$ es el cálculo de impuestos a pagar. Al igual que con la base imponible, se definen
dos variables para las retenciones: $R_{o}$ para la suma de las retenciones que realizan los
empleadores secundarios y $R_{p}$ para el empleador principal.

\begin{enumerate}
	\item $R \in [0, I)$€
	\item $R \in (I, \infty)$€
	\item $R = I$€
	\item $R \notin \mathbb{R}, R < 0$€ (\textcolor{red}{Inválido})
\end{enumerate}

\subsection{Préstamos hipotecarios ($P_{H}$)}
En este caso, se define $P_{H}$ como la cantidad de dinero que se ha pagado
en concepto de préstamos hipotecarios en viviendas habituales anteriores a 2013.

$60266$€ es el valor por encima del cual se dejaría de aplicar la deducción por vivienda habitual,
que resulta del siguiente cálculo: $$\frac{9040}{0.15} = 60266.66666\dots$$

\begin{enumerate}
	\item $P_{H} = 0$€
	\item $P_{H} \in (0, 60266]$€
	\item $P_{H} \in (60266, \infty)$€
	\item $P_{H} \notin \mathbb{R}, P_{H} < 0$€ (\textcolor{red}{Inválido})
\end{enumerate}

\begin{notebox}
	El valor máximo de la deducción es 9040€, lo que no significa que por encima de dicho valor no se aplique
	la deducción, sino que el valor de la deducción no aumenta por encima de dicho valor.
\end{notebox}

\newpage{}
\section{Salidas}
\subsection{Obligatoriedad de la declaración}
\begin{itemize}
	\item Obligatorio
	\item No obligatorio
\end{itemize}

\subsection{Deducciones}
Resume las deducciones que se aplican a la declaración, que en el caso de este enunciado
es solo la deducción por préstamos hipotecarios.

Puesto que solo se tiene en cuenta una deducción, el cálculo debe de caer en el rango
de valores $[0, 9040]$\texteuro.

\subsection{Liquidación final}
El cálculo de la declaración se realiza en base a las retenciones y la base imponible por \textit{tramos}
tal y como define el enunciado, por lo que la salida del cálculo final dependerá de los valores
de las entradas.

Esta salida tiene especial conexión con la entrada $R$, ya que el cálculo de la declaración
depende directamente de la cantidad de retenciones que se han realizado. Al cálculo final de esta salida
deberían aplicarse las deducciones correspondientes (en este enunciado solo se considera $P_{H}$), es
decir, la clase de salida anterior.
