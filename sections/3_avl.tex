\chapter{Análisis de valores límite}
Para el análisis de valores límite, se han seleccionado los valores límite de las clases de equivalencia
de entrada definidas en el capítulo anterior, poniendo especial atención al cumplimiento de las aperturas
de los intervalos.

\section{Base imponible ($BI$)}
Debido a la naturaleza de esta variable, se han seleccionado los valores límite de los intervalos definidos
anteriormente.
\begin{itemize}
	\item $BI = 0$€
	\item $BI = 8999$€
	\item $BI = 9000$€
	\item $BI = 12449$€
	\item $BI = 12450$€
	\item $BI = 20199$€
	\item $BI = 20200$€
	\item $BI = 35199$€
	\item $BI = 35200$€
	\item $BI = 59999$€
	\item $BI = 60000$€
	\item $BI = 299999$€
	\item $BI = 300000$€
\end{itemize}
\newpage{}
\section{Retenciones ($R$)}
Esta variable es especialmente interesante para el AVL, ya que el cálculo de las retenciones depende
directamente de la base imponible, por lo que debería haber una relación directa entre los valores límite
de ambas variables.
\begin{itemize}
	\item $R = 0$€
	\item $R = BI - 1$€
	\item $R = BI$€
	\item $R = BI + 1$€
\end{itemize}

\section{Número de empleadores}
En esta sección no se considera relevante aplicar técnicas AVL, ya que el mero cumplimiento de las
diferentes clases de equivalencia garantiza el correcto funcionamiento del sistema debido a la pertenencia de
los posibles valores al conjunto entero.

\section{Préstamos hipotecarios}
\begin{itemize}
	\item $P_{H} = 0$€
	\item $P_{H} = 60266$€
	\item $P_{H} = 60267$€
\end{itemize}
